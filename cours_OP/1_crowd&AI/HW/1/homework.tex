\documentclass[14pt]
{article}

%basiques
\usepackage[utf8x]{inputenc}
\usepackage[T1]{fontenc}
\usepackage[english]{babel}

%math
\usepackage{amsmath}
\usepackage{amssymb}

% images
\usepackage{graphicx}

% links and refs
\usepackage{hyperref}
\hypersetup{
    colorlinks = true,
    linkcolor  = {red}.
}


%mise en page
\usepackage{fancyhdr}
\usepackage{fancybox}
\usepackage{geometry}

\newcommand\tab[1][1cm]{\hspace*{#1}}

\geometry{hmargin=4cm}

\begin{document}
% Entete
\pagestyle{fancy}
\lhead{Léa Heiniger}
\rhead{08.03.2022}
\chead{\textbf{Crowdsourcing and AI}}

\bigskip
\begin{center}
	\section*{\textbf{{\LARGE Homework : "The Rise of Crowdsourcing"}}}
\end{center}
\bigskip\bigskip\bigskip

\paragraph*{\\}\textbf{Type of crowdsourcing mentioned in the article \href{https://www.wired.com/2006/06/crowds/}{The Rise of Crowdsourcing} :}\\
\begin{itemize}
\item[1)] Platforms to sell what you do :\\
\begin{itemize}
\item contributive (photo) database (iStockphoto) \\
\item selling place for particulars (eBay,...) \\
\end{itemize}

\item[2)] Creating new content out of things that already exists (or that especially created by the crowd) : \\
\begin{itemize}
\item show using internet videos\\
\end{itemize}

\item[3)] Helping the research : \\
\begin{itemize}
\item distributed computing projects (SETI@home) \\
\item scientific problems solving (InnoCentive) \\ 
\item perform tasks that computers can't do or are bad at doing (Amazon's Mechanical Turk)\\
\end{itemize}

\item[3 bis)] Exchange of information/resources with every one:\\
\begin{itemize}
\item exchange of knowledge (Wikipedia)\\
\item open source software \\
\end{itemize}
\end{itemize}

\bigskip
\paragraph*{}Almost all the crowdsourcing project that are presented in this article are using money as a motivation for the crowd to participate.

\end{document}