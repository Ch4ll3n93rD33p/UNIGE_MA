\documentclass[14pt]
{article}

%basiques
\usepackage[utf8x]{inputenc}
\usepackage[T1]{fontenc}
\usepackage[english]{babel}

%math
\usepackage{amsmath}
\usepackage{amssymb}

% images
\usepackage{graphicx}

% links and refs
\usepackage{hyperref}
\hypersetup{
    colorlinks = true,
    linkcolor  = {red}.
}


%mise en page
\usepackage{fancyhdr}
\usepackage{fancybox}
\usepackage{geometry}

\newcommand\tab[1][1cm]{\hspace*{#1}}

\geometry{hmargin=4cm}

\begin{document}
% Entete
\pagestyle{fancy}
\lhead{Léa Heiniger}
\rhead{15.03.2022}
\chead{\textbf{Crowdsourcing and AI}}

\bigskip
\begin{center}
	\section*{\textbf{{\LARGE Homework : "Crowd science user contribution patterns and their implications"}}}
\end{center}
\bigskip\bigskip\bigskip
%%%%%%%%%%%%%%%%%%%%%
\paragraph*{\\}\textbf{Type of crowdsourcing mentioned in the article \href{https://www.wired.com/2006/06/crowds/}{The Rise of Crowdsourcing} :}\\
What challenges do the authors identify for crowd science?


How do the results inform current science policy discussions?


What aspects of their analysis were not clear to you, and why?

\end{document}