\documentclass[14pt]
{article}

%basiques
\usepackage[utf8x]{inputenc}
\usepackage[T1]{fontenc}
\usepackage[english]{babel}

%math
\usepackage{amsmath}
\usepackage{amssymb}

% images
\usepackage{graphicx}

% links and refs
\usepackage{hyperref}
\hypersetup{
    colorlinks = true,
    linkcolor  = {red}.
}


%mise en page
\usepackage{fancyhdr}
\usepackage{fancybox}
\usepackage{geometry}

\newcommand\tab[1][1cm]{\hspace*{#1}}

\geometry{hmargin=4cm}

\begin{document}
% Entete
\pagestyle{fancy}
\lhead{Joao Costa da Quinta, Léa Heiniger}
\rhead{15.03.2022}
\chead{\textbf{\tab \tab Crowdsourcing and AI}}

\bigskip
\begin{center}
	\section*{\textbf{{\LARGE BOINC}}}
\end{center}
\bigskip\bigskip\bigskip

\paragraph*{}We looked at the \href{https://escatter11.fullerton.edu/nfs/}{NFS@home} project. The objective is to use computer resources to factorise very large numbers using the number field sieving algorithm.\\

\section*{BOINC :}
\paragraph*{} BOINC is licensed under the GNU Free Documentation License, which allows users to copy and distribute the program but not to modify it. They probably chose this license to encourage people to share the software with others while being sure that no one can freely modify the software.\\

\section*{Science :}
\paragraph*{} We learned about the RSA algorithm and how its security depends on our ability to factor numbers..\\

\section*{Technology :}
\paragraph*{} The BOINC interface is not user-friendly, not all windows are well displayed and it is not intuitive to know where to find this or that information. The project website is not very intuitive either.\\

\section*{Society :}
\paragraph*{}  We have not found the exact number of participants but in the user ranking we can estimate the number of users. There are currently ~1000 active users and over 9000 participants since the start of the project. \\



\end{document}