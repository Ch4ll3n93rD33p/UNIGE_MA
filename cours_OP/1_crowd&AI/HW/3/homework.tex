\documentclass[14pt]
{article}

%basiques
\usepackage[utf8x]{inputenc}
\usepackage[T1]{fontenc}
\usepackage[english]{babel}

%math
\usepackage{amsmath}
\usepackage{amssymb}

% images
\usepackage{graphicx}

% links and refs
\usepackage{hyperref}
\hypersetup{
    colorlinks = true,
    linkcolor  = {red}.
}


%mise en page
\usepackage{fancyhdr}
\usepackage{fancybox}
\usepackage{geometry}

\newcommand\tab[1][1cm]{\hspace*{#1}}

\geometry{hmargin=4cm}

\begin{document}
% Entete
\pagestyle{fancy}
\lhead{Léa Heiniger}
\rhead{22.03.2022}
\chead{\textbf{Crowdsourcing and AI}}

\bigskip
\begin{center}
	\section*{\textbf{{\LARGE Homework : Reinventing Discovery}}}
\end{center}
\bigskip\bigskip\bigskip

\section*{1. What example of Open Source innovation does the author give, that is inspired by open source software?}
\paragraph*{} The author give examples in various domains such as architecture (Open Architecture Network), Digital Art, Biology, Encyclopedia (Wikipedia) and sharing ideas (Polymath Project)\\

\section*{2. What unique features of open source collaboration does the author distinguish?}
\paragraph*{} The author speaks about four patterns in open source collaboration:\\
\begin{itemize}
\item Working in a modular way and subdividing tasks\\
\item Favouring small contributions \\
\item Authorise to reuse of previous work (done by other people)\\
\item Help people decide where to direct their attention using signalling mechanisms
\end{itemize} 

\end{document}